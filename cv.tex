\documentclass[a4paper,12pt]{article}

\usepackage[english]{babel}
\usepackage[table]{xcolor}
\usepackage{graphicx}
\usepackage[T1]{fontenc}
\usepackage{multicol}
%\usepackage{multirow}
\usepackage[tableposition=top]{caption}
\usepackage{subcaption}
	\DeclareCaptionStyle{MyCaption}{labelfont={small,sf,bf},textfont={small}}
	\captionsetup{style=MyCaption}
	\captionsetup[subtable]{list=true, style=MyCaption}
	
\usepackage{geometry}
	\geometry{
			outer=23mm,
			inner=23mm,
			top=20mm,
			bottom=15mm}

\definecolor{light-gray}{gray}{0.77}	
\arrayrulecolor{light-gray}
\setlength{\arrayrulewidth}{1pt}


\begin{document}
\begin{titlepage}

\flushright\Huge\textbf{{Curriculum Vitae}}

\vspace{1.3cm}
\flushleft\large Personal Data
\\[-2ex]
\rule{1\textwidth}{0.7pt}
\smallskip

\begin{table}[h!]
\begin{subtable}[t]{10.5cm}
	\begin{tabular}{p{3.7cm}p{0.5cm}|p{12cm}}
	\small
	Name & & Julia Kukulies\\
	& &\\[-1.7ex]
	Adress & & Gr{\"a}ddgatan 5
	\\ 	& & 41276  Gothenburg\\
	& & Sweden\\
	& & \\[-1.7ex]
	Date of Birth & & 21.11.1992\\
	& & \\[-1.7ex]
	Birthplace & & Kiel, Germany\\
	& & \\[-1.7ex]
	E-Mail & & julia.kukulies@gu.se\\
	\end{tabular}
\end{subtable}
\hfill
\begin{subtable}{5.5cm}
\includegraphics[width=1.\textwidth]{portrait.jpg}
\end{subtable}
\end{table}

\flushleft\large Education
\\[-2ex]
\rule{1\textwidth}{0.7pt}
\smallskip



\begin{table}[h!]
	\begin{tabular}{p{3.7cm}p{0.5cm}|p{12cm}}
    Sep 2018 -- & & University of Gothenburg, Sweden \\
      & & Department of Earth Sciences, \textit{Regional Climate Group} \\
  & & PhD candidate in Natural Sciences \\
   && \\
    & & Title of PhD project: \textit{Observing and Modeling Convective Systems and their Role in the Water Cycle in the Tibetan Plateau region}\\
	& & \\[-1.7ex]
    &&\\ 
  Sep 2017 -- Jun 2018 & & University of Gothenburg, Sweden \\
  & & M.Sc. in Atmospheric Sciences\\
  &&Final grade: VG (Excellent)\\
  && \\
	& & \\[-1.7ex]
    &&\\
  Sep 2017 -- Dec 2017 && Nanjing University, China \\
  && School of Atmospheric Sciences: Aerosol-cloud Research Group  \\
  && Research visit\\
  &&\\
	& & \\[-1.7ex]
	Sep 2013 -- Jun 2016 & & University of Gothenburg, Sweden\\
	& & B.Sc. in Earth Sciences with Major in Climatology\\
  &&Final grade: VG (Excellent)\\
	\end{tabular}
\end{table}
\newpage



\flushleft\large Extracurricular Activities
\\[-2ex]
\rule{1\textwidth}{0.7pt}
\smallskip
\begin{table}[h!]
  \begin{tabular}{p{3.7cm}p{0.5cm}|p{12cm}}
 Sep 2018 - Jun 2021 && PhD representative and Board member of GAC \\
  && (Gothenburg Air and Climate Network)\\
  &&\\
 Sep 2018 - Jun 2021 && Executive Secretary of APECS \\
  && (Association of Polar and Alpine Early Career Scientists)\\
  &&\\
	Jul 2016 -- Sep 2016 & & Internship at Max Planck Institute for Meteorology \\
	%& & Working group: Terrestial Hydrology\\
  && Project title: \textit{Validation of a global dynamical wetland scheme in land-atmosphere coupled and uncoupled simulations}\\
	& & Hamburg, Germany \\
  &&\\
	Sep 2015 -- Jan 2016 & & Student course assistent in \textit{Geographical Information Sciences}\\
  &&  University of Gothenburg, Sweden\\
  &&\\
	Jun 2014 -- Aug 2014 & & Student research assistent at Helmholtz Centre for Ocean Research \\
	&& Kiel, Germany \\
	& & \\[-1.7ex]
	\end{tabular}
\end{table}


\flushleft\large International Research Schools 
\\[-2ex]
\rule{1\textwidth}{0.7pt}
\smallskip
\begin{table}[h!]
  \begin{tabular}{p{3.7cm}p{0.5cm}|p{12.5cm}}
 Jan 2020 && European Research School on Atmospheres\\
  && Grenoble, France\\
 &&\\
 Sep 2019 &&  Max Planck Research School on Earth System Modeling \\
  && Hamburg, Germany\\
 &&\\
 Oct 2018 && NEGI Course on Escience tools for climate research\\
 &&\textit{Climate modeling and model evaluation at high latitudes}\\
 && Andoya, Norway\\

 Mar 2019 && Arctic in a Changing Climate (ClimbEco course)\\
 && Gothenburg, Sweden\\
 &&\\
 Aug 2018 && Summer school on Air quality in China\\
 && Helsinki, Finland \\
 &&\\
 Jun 2018 && ITPCAS Summer School on Climate Modeling\\
 && Beijing, China\\
 &&\\
 Sep 2016 &&  Baltic Earth Summer School on \textit{Atmosphere-Ocean Climate Models} \\
  && Ask\"{o}, Sweden\\
 &&\\

& \\[-1.7ex]
	\end{tabular}
\end{table}

\newpage

\flushleft\large Teaching
\rule{1\textwidth}{0.7pt}
\begin{table}[h!]
  \begin{tabular}{p{3.7cm}p{0.5cm}|p{12.5cm}}
 2019 - 2021 && \textbf{Teaching assistant} in Climate change and society\\
  && University of Gothenburg, Sweden\\
 &&\\
 2021 &&  \textbf{Teaching assistant} in Advanced Climate data analysis \\
  && University of Gothenburg, Sweden\\
 &&\\
 2020 && \textbf{Teaching assistant} in Climate Modeling\\
 && University of Gothenburg, Sweden\\
 &&\\
 2019 && \textbf{Teaching assistant} in Applied Climatology\\
 && University of Gothenburg, Sweden\\\
& \\[-1.7ex]
	\end{tabular}
\end{table}



\flushleft\large Scholarships and funds
\\[-1ex]
\rule{1\textwidth}{0.7pt}
\begin{table}[h!]
  \begin{tabular}{p{3.7cm}p{0.5cm}|p{12.5cm}}
 2021 && NCAR Advanced Study Programme (Colorado, US)\\
 &&\\
 2019 && Research fund Adlerbertska stiftelse (Sweden) \\
 &&\\
 2018 && Research fund Sven Lindqvists forskningsstiftelse(Sweden)\\
& \\[-1.7ex]
	\end{tabular}
\end{table}


\flushleft\large Conference presentations
\\[-1ex]
\rule{1\textwidth}{0.7pt}
\begin{table}[h!]
  \begin{tabular}{p{3.7cm}p{0.5cm}|p{12.5cm}}
 2021 && Meso-scale weather systems and their interaction in the Tibetan Plateau region (vPico talk), European Geoscience Union, online\\
 &&\\
 2019 && Spatial and temporal dynamics of convective precipitation cells on the Tibetan Plateau (poster), International Conference on Regional
Climate-CORDEX 2019, China \\
&&\\
 2019 && Spatial and temporal dynamics of convective precipitation cells on the Tibetan Plateau (oral presentation), European Meteorological Society, Denmark \\
 &&\\
 2018 && Temporal and Spatial variations of clouds and convection over the Tibetan Plateau derived from CloudSat satellite retrievals (oral presentation), 8th Third Pole Environment workshop, Sweden\\
& \\[-1.7ex]
	\end{tabular}
\end{table}


\newpage


\flushtop
\flushleft\large Publications
\\[-1ex]
\rule{1\textwidth}{0.7pt}
%\end{titlepage}
\begin{table}[h!]

Lai, H. W., Chen, H. W., \textbf{Kukulies, J.}, Ou, T. and  Chen, D. (2020). Regionalization of seasonal precipitation over the Tibetan Plateau and associated large-scale atmospheric systems. \textit{Journal of Climatology.}\\
\\[-1.7ex]


    \textbf{Kukulies, J.}, Chen, D. and Wang, M. (2020). Temporal and spatial variations of convection and precipitation over the Tibetan Plateau based on recent satellite observations. Part II: Precipitation climatology derived from GPM. \textit{International Journal of Climatology.}\\

        \\[-1.7ex]


    \textbf{Kukulies, J.}, Chen, D. and Wang, M. (2019). Temporal and spatial variations of convection and precipitation over the Tibetan Plateau based on recent satellite observations. Part I: Cloud climatology derived from CloudSat and CALIPSO. \textit{International Journal of Climatology.}\\
    \\[-1.7ex]

\end{table}


\flushtop
\flushleft\large Other Qualifications
\\[-2ex]
\rule{1\textwidth}{0.7pt}
\begin{table}[h!]
	\begin{tabular}{p{3.7cm}p{1cm}|p{11cm}}
		Programming & &  Python (\textit{Advanced}), Linux/ Bash scripting (\textit{Good}), R (\textit{Basic}), MATLAB (\textit{Basic})\\
		& & \\	[-2.5ex]
		Languages & & German (\textit{Mothertongue}), English (\textit{Fluent}), Swedish (\textit{Fluent}), French (\textit{Good}), Spanish (\textit{Beginner}) \\
	\end{tabular}
\end{table}



%\flushleft\large References
%\\[-2ex]
%\rule{1\textwidth}{0.7pt}
%\end{titlepage}
%\begin{table}[h!]
%	\begin{tabular}{p{3.7cm}p{1cm}|p{11cm}}

%		Deliang Chen  & & \textit{Professor} \\

 %   Minghuai Wang & & \textit{Professor}\\

	%	& & \\[-1.7ex]
%		Tobias Stacke  & &  \textit{Doctor}\\
		
%	\end{tabular}
%\end{table}

%\vfill
\flushright\small Last updated: G\"{o}teborg, \today

\end{titlepage}
\end{document}













